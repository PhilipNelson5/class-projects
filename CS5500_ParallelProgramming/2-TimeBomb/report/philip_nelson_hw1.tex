\documentclass{article}

\usepackage[utf8]{inputenc}
\usepackage{geometry}
\usepackage{listings}
\usepackage{courier}

\title{HW 1 \\ Get MPI working on your computer}
\author{Philip Nelson}
\date{2018 September 7}

\lstset{basicstyle=\footnotesize\ttfamily\normalsize,
        breaklines=true,
        stepnumber=1,
       }

\begin{document}

\maketitle

\section*{Installation}

I am currently using Manjaro, a Linux distro based on Arch, on which MPI is already installed.

\section*{Code}

\begin{lstlisting}[showstringspaces=false, language=c++, numbers=left]
#include <iostream>
#include <mpi.h>

#define MCW MPI_COMM_WORLD

int main(int argc, char** argv)
{
  int rank, size, data;

  MPI_Init(&argc, &argv);
  MPI_Comm_rank(MCW, &rank);
  MPI_Comm_size(MCW, &size);

  MPI_Send(&rank, 1, MPI_INT, (rank + 1) % size, 0, MCW);
  MPI_Recv(&data, 1, MPI_INT, MPI_ANY_SOURCE, 0, MCW, MPI_STATUS_IGNORE);

  std::cout << "I am process " << rank << " of " << size
            << " and received a message from process " << data
            << std::endl;

  MPI_Finalize();

  return EXIT_SUCCESS;
}
\end{lstlisting}

\section*{Output}

\begin{lstlisting}[showstringspaces=false]

# mpiexec -n 4 release

I am process 2 of 4 and received a message from process 1
I am process 3 of 4 and received a message from process 2
I am process 0 of 4 and received a message from process 3
I am process 1 of 4 and received a message from process 0

\end{lstlisting}


\end{document}
