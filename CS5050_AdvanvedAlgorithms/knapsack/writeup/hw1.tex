\documentclass[11pt]{article}

%\usepackage{latexsym}
%\newcommand{\epsfig}{\psfig}
%\usepackage{tabularx,booktabs,multirow,delarray,array}
%\usepackage{graphicx,amssymb,amsmath,amssymb,mathrsfs}
%\usepackage{hyperref}
%\usepackage[linesnumbered, vlined, ruled]{algorithm2e}

%\usepackage{fullpage}
\usepackage[top=1in, bottom=1in, left=1in, right=1in]{geometry}
\usepackage{amsmath}
\usepackage{listings}

\begin{document}
\baselineskip=14.0pt

\title{CS5050 \textsc{Advanced Algorithms}
\\{\Large Philip Nelson: worked with Raul Ramirez, Ammon Hepworth}
\\ Assignment 1: Algorithm Analysis
\\ {\large {\bf Due Date:} {\bf 3:00 p.m.}, Thursday, Jan. 25, 2018 ({\bf at the beginning of CS5050 class})}}
\date{}
%\date{\today}

\maketitle

\vspace{-0.5in}

\begin{enumerate}

\item
{ \bf (10 points)}
This exercise is to convince you that exponential time algorithms are not quite useful.

\begin{enumerate}

\item
For the input size $n=100$:

\[\frac{2^{100}}{1.25\cdot 10^{17}\cdot 3600 \cdot 24 \cdot 365 \cdot 100} = 3215.75 \text{ centuries}\]

\item
For the input size $n=1000$:

\[\frac{2^{1000}}{1.25\cdot10^{17}\cdot 3600 \cdot 24 \cdot 365 \cdot 100} = 2.72\cdot 10^{274} \text{ centuries}\]
\end{enumerate}

{\bf Note:} You may assume that a year has exactly 365 days.

\item
{\bf (20 points)}
Order the following list of functions in increasing order asymptotically (i.e., from small to large, as we did in class).


\begin{table}[h]
\begin{center}
\begin{tabular}{ccccccc}
	$\log n$ & $n!$ & $2^{500}$ &  $2^n$ & $\log(\log n)^2$
					& $2^{\log n}$\\
$\log^3n$  & $n\log n$   &  $\log_4n$   & $n^3$   &
$\sqrt{n}$  & $n^2\log^5 n$  \\
\end{tabular}
\end{center}
\end{table}

\vspace{-0.4in}

\[2^{500} < \log(\log n)^2 < \log n \leq \log_4 n < log^3 n < \sqrt{n} < 2^{\log n} < n\log n < n^2 \log^5 n < n^3 < 2^n < n!\]
\item
{\bf (30 points)}
For each of the following pairs of functions, indicate
whether it is one of the three cases: $f(n)=O(g(n))$, $f(n)=\Omega(g(n))$, or $f(n)=\Theta(g(n))$. For each pair, you only need to give your answer and the proof is not required.


\begin{enumerate}
\item $f(n)=7\log n$ and $g(n)=\log n^3 + 56$.\hfill $f(n)=\Theta(g(n))$
\item $f(n)=n^2+n\log^3 n$ and $g(n)=6n^3+\log^2n$.\hfill $f(n)=O(g(n))$
\item $f(n)=5^n$ and $g(n)=n^22^n$.\hfill $f(n)=\Omega(g(n))$
\item $f(n)=n\log^2n$ and $g(n)=\frac{n^2}{\log^3 n}$.\hfill $f(n)=O(g(n))$
\item $f(n)=\sqrt{n}\log n$ and $g(n)=\log^8n+25$.\hfill $f(n)=\Omega(g(n))$
\item $f(n)= n\log n+6n$ and $g(n) = n\log_3 n-8n$.\hfill $f(n)=\Theta(g(n))$
\end{enumerate}



\newpage
\item
{\bf (20 points)}
This is a ``warm-up'' exercise on algorithm {\bf design} and {\bf analysis}.

\begin{description}

\item[1. Algorithm Description]

  The \textit{fill\_knapsack} algorithm can fill a knapsack of size $K$ from an array $A$ to at least $\frac{K}{2}$ in a single linear scan. During the linear scan, check each element $a_i : \frac{K}{2} \leq a_i \leq K$. If an $a_i$ is found, immediately return it as a single element solution. If the element is not a single element solution, check if it is an $a_j : a_j < \frac{K}{2}$. Add elements $a_j$ to the knapsack and check if it is at least $\frac{K}{2}$ full. If the knapsack is at least $\frac{K}{2}$ full, return it.
  %If no single element solution is found, the solution will be a subset of the elements $a_j < \frac{K}{2}$ such that $\frac{K}{2} \leq \sum_{i =1}^{|S|} x_i \leq K$.

\vspace{0.1in}
\item[2. Pseudocode]
fill\_knapsack

\begin{lstlisting}[language=C++,
                   % directivestyle={\color{black}}
                   % emph={int,char,double,float,unsigned},
                   % emphstyle={\color{blue}}
                  ]
array fill_knapsack(array A, K)
{
  array knap;
  int sum = 0;

  for (i = 0; A.size(); ++i) // single linear scan
  {
    if (K / 2 <= A[i] && A[i] <= K) // check for one element solution
      return {A[i]};

    if (A[i] < K / 2) // all elements < K/2
    {
      sum += A[i];
      knap.push_back(A[i]);

      if (sum >= K/2) // return knapsack when it is more than K/2 full
        return knap;
} } }
\end{lstlisting}

\item[3. Correctness]

  The first way to fill the knapsack is obviously correct. The algorithm will put any single element $a_i : \frac{K}{2} \leq a_i \leq K$ in the knapsack. If no such single element exists, the knapsack will be filled with elements $a_j : a_j < \frac{K}{2}$. It is impossibe to overfill the knapsack using this method. If two elements $a_1$ and $a_2$ are put in the knapsack such that $a_1, a_2 < \frac{K}{2}$, the elements will be $\frac{k-1}{2}$ at the largest. Thus

  \[a_1 + a_2 = \frac{K-1}{2}+\frac{K-1}{2}=\frac{2K-2}{2}=K-1\]

  Therefore any $a_1 + a_2$, where $a_1$ and $a_2$ could be combinations of multiple $a_j$, will never be more than $K-1$ therefore a solution will always be found.

\vspace{0.1in}
\item[4. Time Analysis]
Please make sure that you analyze the running time of your algorithm.

The algorithm \textit{fill\_knapsack} solves the knapsack problem, factor two approximation, in order $O(n)$. The order is $O(n)$ because there is one for loop which does a linear scan over the array $A$ containing $n$ elements. During the linear scan, a constant amount of work is done; this only affects the order by a constant ammount and therefore can be ignored. Thus, \textit{fill\_knapsack} is $O(n)$.

\end{description}
\end{enumerate}


\vspace{0.2in}
{\bf Total Points: 80} (not including the five bonus points)

\end{document}
