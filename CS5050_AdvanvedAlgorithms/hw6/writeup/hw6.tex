\documentclass[11pt]{article}

\usepackage[linesnumbered, vlined, ruled]{algorithm2e}
\usepackage{fullpage}
\usepackage{graphicx}
\usepackage{latexsym}
\usepackage{tcolorbox}
\usepackage{amsmath}

\evensidemargin=0.0in
\headheight=0.0in
\oddsidemargin=0.0in
\textheight=9.0in %9.1in
\textwidth=6.5in   %6.55in
\topmargin=-0.40in %0.35
\def\report{{\em Report-Max}$(1)$}
\def\reportk{{\em Report-Max}$(k)$}

\tcbset{
  colbacktitle = black!50!white,
  colback = white,
  title = {\bf Solution},
}

\begin{document}

\baselineskip=14.0pt

\title{CS5050 \textsc{Advanced Algorithms}
\\ {\bf Philip Nelson}
\\ {\Large Spring Semester, 2018}
\\ Assignment 6: Graph Algorithms I
\\ {\large Worked with: Ammon H, Raul R, Hailee M, Kyle H, and Jack K}}
\date{}

\maketitle

\vspace{-0.6in}

{\bf Note:} In this assignment, we assume that all input graphs are represented by adjacency lists.

\begin{enumerate}
  \item % 1)

We say that a {\bf directed} graph $G$ is {\em strongly connected} if for every pair of vertices $u$ and $v$, there exists a path from $u$ to $v$ and there also exists a path from $v$ to $u$ in $G$.

Given a directed graph $G$ of $n$ vertices and $m$ edges, let $s$ be a vertex of $G$.

\begin{enumerate}
  \item % a)
Design an $O(m+n)$ time algorithm to determine whether the following is true: there exists a path from $s$ to $v$ in $G$ for all vertices $v$ of $G$. {\hfill \bf (10 points)}

\begin{tcolorbox}

  To determine if there is a path from $s$ to $v$ for all $v$ in $G$ we can use
  a breadth first search. As nodes are visited, keep a count of how many are
  reachable from $s$. If the count is equal to the number of nodes in $G$ then
  return true, else return false.\\

  A breadth first search can be accomplished in $O(m+n)$

\end{tcolorbox}

  \item % b)
Design an $O(m+n)$ time algorithm to determine whether the following is true: there exists a path from $v$ to $s$ in $G$ for all vertices $v$ of $G$. {\hfill \bf (10 points)}

\begin{tcolorbox}

  The method to determine whether there exists a path from all $v$ in $G$ to $s$
  is similar to the previous except we want to perform a breadth first search
  following the reverse direction of all the edges.\\

  In order to be able to follow the edges in the reverse direction, we need to
  construct the reverse adjacency list. This can be done by iterating through
  the adjacency list of each vertex and adding the vertex $v$ to the new list of
  each vertex in its list. For example, if the list for vertex $A$ is
  $[B \rightarrow D \rightarrow E]$, then add $A$ to the reverse adjacency lists
  of $B,\ D,$ and $E$. This can be done in $O(m+n)$.\\

  Once the reverse adjacency list is constructed we can use it to perform a
  breadth first search as described in part $a$. This takes $O(m+n)$. Together
  with construing the reverse adjacency list, the time complexity is
  $O(2(m+n)) \Rightarrow O(m+n)$.

\end{tcolorbox}

  \item % c)
Prove: {\em $G$ is strongly connected if and only if there exists a path in $G$ from $s$ to $v$ and there is also a path from $v$ to $s$  for all vertices $v$ of $G$}.{\hfill \bf (10 points)}

\begin{tcolorbox}[title={\bf Proof}]

    If there exists a path in $G$ from $s$ to $v$ and from
    $v$ to $s$ for all vertices $v$ of $G$, then $G$ is strongly connected.
    This is true because of the transitive property;
    if you can go from any vertex $v$ to $s$, and from $s$ you can go to any other
    vertex $u$, then you can always go from $v$ to $u$ through $s$ and vice versa.
    In this fashion, you can start from any vertex and visit every other vertex
    which makes $G$ strongly connected.

  \end{tcolorbox}
\end{enumerate}

  \item % 2)
Given an undirected graph $G$ of $n$ vertices and $m$ edges, suppose $s$ and $t$ are two vertices in $G$. We have already known that we can find a shortest path from $s$ to $t$ by using the breadth-first-search (BFS) algorithm. However, there might be multiple different shortest paths from $s$ to $t$.

Design an $O(m+n)$ time algorithm to compute the number of different shortest paths from $s$ to $t$ (your algorithm does not need to list all the shortest paths, but only report their number). (Hint: Modify the BFS algorithm.)
{\hfill \bf (20 points)}

\begin{tcolorbox}

  To find the number of shortest paths from $s$ to $t$, we will use a modified
  breadth first search. Begin by setting the path length at each node to $\infty$
  and the path length at $s$ to $0$. We need another point of data, the
  number of paths that go through the node, set this value to zero, for all
  vertices and one for $s$. Then begin a breadth first search from $s$.
  When visiting a node for the first time, enqueue
  its unvisited neighbors. Update the path length of the neighbors to the current node's path
  length $+\ 1$, and add the number of paths from the vertex to the number of paths in the neighbor.
  However, when a node has already been visited, if the path length $+\ 1$ is
  equal to the neighboring nodes path length, just add the number of paths
  in the vertex to the paths in the neighbor. Since the paths have equal length through the vertex,
  it would not matter which pat was chosen, they are equivalent. When the queue is empty,
  the number of paths from $s$ to $t$ will be stored in $t$.\\

  Since breadth first search can be accomplished in $O(m + n)$ therefore the time
  complexity is $O(m+n)$.

\end{tcolorbox}

\newpage
  \item % 3)
Given a directed-acyclic-graph (DAG) $G$ of $n$ vertices and $m$ edges,
let $s$ and $t$ be two vertices of $G$. There might be multiple different paths (not necessarily shortest paths) from $s$ to $t$.
Design an $O(m+n)$ time algorithm to compute the number of different paths in $G$ from $s$ to $t$. {\hfill \bf (20 points)}

\begin{tcolorbox}

  To count the paths from $s$ to $t$ in a DAG, we need to follow a topological
  order for $G$. Then we use a similar algorithm as described in problem 2.
  Label all nodes with the number of paths that pass through them, initialized to
  zero, with $s$ starting at one. Then, beginning with $s$, add the paths from $s$
  to the paths of $s$'s adjacent neighbors. Then repeat the process with the next
  node in the list until $t$ is processed. $t$ will contain the number of paths
  to it from $s$.\\

  Following the topological order, all nodes and edges are used once,
  therefore the time complexity is $O(m+n)$

\end{tcolorbox}

  \item % 4)
In class we have studied an algorithm for computing shortest paths in DAGs.
Particularly, the {\em length} of a path is defined to be the total sum of the
weights of all edges in the path. Sometimes we are not only interested in a
shortest path but also care about the number of edges of the path.

Let $G$ be a DAG (directed-acyclic-graph) of $n$ vertices and $m$ edges. Each
edge $(u,v)$ of $E$ has a weight $w(u,v)$ (which can be positive, zero, or
negative). Let $k$ be a positive integer, which is given in the input. A path
in $G$ is called a {\em $k$-link} path if the path has {\bf no more than} $k$
edges. Let $s$ and $t$ be two vertices of $G$. A {\em $k$-link shortest path}
from $s$ to $t$ is defined as a $k$-link path from $s$ to $t$ that has the
minimum total sum of edge weights among all possible $k$-link $s$-to-$t$ paths
in $G$.

Design an $O(k(m+n))$ time algorithm to compute a $k$-link shortest path from $
s$ to $t$. For simplicity, you only need to return the length of the path and
do not need to report the actual path.
{\hfill \bf (20 points)}

\begin{tcolorbox}

  To find the {\em $k$-link shortest path} from $s$ to $t$ we can find
  the shortest paths of edge count $e$ s.t. $1 \leq e \leq k$ using topological
  ordering. Once the shortest paths are known, take the shortest one.
  Since it takes $O(m+n)$ time to find the shortest path of length $e$
  and we will need to perform this operation $k$ times, the complexity is $O(k(m+n))$.\\

  To find the shortest path of edge length $e$, we will use the topological ordering
  and two data points, path length and edge number length. Initialize them to $\infty$
  for all $v$ and for $s$, $0$ and $1$ respectively.
  Starting with $s$, push the path length plus the adjoining edge weight from
  $s$ to the adjacent vertices of $s$ if the edge number length $+\ 1$ is
  less than or equal to the edge number length at the adjacent vertex,
  and the path length plus the adjoining edge length is less than the path length
  is less than the path length in the vertex. If at any point the edge length
  number is greater than $e$, stop pursuing the path. At the end, if there
  exists a path with $e$ edges, its length will be stored in $t$.

\end{tcolorbox}


\end{enumerate}

\end{document}
