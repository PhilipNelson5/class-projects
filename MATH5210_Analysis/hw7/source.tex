\documentclass[10pt,letterpaper]{article}

\usepackage{amsmath}
\usepackage{amsthm}
\usepackage{amssymb}
\usepackage{amsfonts}
\usepackage{array}
\usepackage{dsfont}

\newcommand\Q{\mathds{Q}}
\newcommand\Z{\mathds{Z}}
\newcommand\R{\mathds{R}}
\newcommand\C{\mathds{C}}
\newcommand\A{\mathds{A}}
\newcommand\ds{\displaystyle}

\newcommand{\imp}{\Rightarrow}
\newcommand\qedsym{\hfill \rule{2mm}{2mm}}

\parindent=0pt
\begin{document}

\title{MATH5210 \textsc{Analysis}
  \\ Assignment 7
  \\ Uniform Continuity
  \\ Philip Nelson
}

\date{}

\maketitle
For Problems 1 to 6, give $\epsilon - \delta$ proofs of uniform continuity.
\section*{1}

\textbf{Claim:} $f(x) = x^2 + 2x -3$ is uniformly continuous on the interval $[2,4]$

\medskip

\textbf{Proof:} Let $\epsilon > 0$ be given. There exists $\delta > 0$ s.t.\ for any $x, y \in [2,4]$ \[|x-y| < \delta \Rightarrow |f(x) - f(y)| < \epsilon\]

Choose $\delta = \frac{\epsilon}{10}$. Then

\[|f(x)-f(y)| = |x^2 + 2x - 3 - y^2 - 2y + 3|\]

\[= |x^2 + 2x - y^2 - 2y|\]

\[= |x^2 - y^2 + 2x - 2y|\]

\[= |(x-y)(x+y) + 2(x-y)|\]

\[= |\delta(x+y) + 2\delta|\]

\[= \delta|x + y + 2|\]

$x + y + 2$ is an increasing function so we need to bound $x$ and $y$ above and we can easily do this by examining the domain. An increasing function achieves it's maximum at the upperboudn of it's domain so $x,y < 4$ therefore

\[<\delta|4+4+2|\]

\[=10\delta = \epsilon\]

Therefore $f(x) = x^2 + 2x -3$ is uniformly continuous on the interval $[2,4]$.

\qedsym

\section*{2}

\textbf{Claim:} $f(x) = x^2 + 2x -3$ is uniformly continuous on the interval $[0,10]$

\medskip

\textbf{Proof:}

\section*{3}

\textbf{Claim:} $g(x) = \frac{1}{x+1}$ is uniformly continuous on the interval $[0,5]$

\medskip

\textbf{Proof:} Let $\epsilon > 0$ be given. There exists $\delta > 0$ s.t.\ for any $x, y \in [0,5]$ \[|x-y| < \delta \Rightarrow |f(x) - f(y)| < \epsilon\]

Choose $\delta = \epsilon$ then

\[|f(x) - f(y)| = \left|\frac{1}{x+1} - \frac{1}{y+1}\right|\]

\[ = \left|\frac{y+1-x-1}{(x+1)(y+1)}\right|\]

\[ = \left|\frac{x-y}{(x+1)(y+1)}\right|\]

\[ = \delta\left|\frac{1}{(x+1)(y+1)}\right|\]

$g(x)$ is a decreasing function so we need to bound $x$ and $y$ below and we can easily do this by examining the domain of $g(x)$ which is $[0,5]$. So $x,y > 0$ therefore

\[ < \delta\left|\frac{1}{(0+1)(0+1)}\right|\]

\[ = \delta\left|\frac{1}{1}\right| = \epsilon\]

Therefore $g(x) = \frac{1}{x+1}$ is uniformly continuous on the interval $[0,5]$.

\qedsym

\section*{4}

\textbf{Claim:} $g(x) = \frac{1}{x+1}$ is uniformly continuous on the interval $[0,\infty)$

\medskip

\textbf{Proof:}

\section*{5}

\textbf{Claim:} $h(x) = \frac{x}{x+1}$ is uniformly continuous on the interval $[0, \infty)$

\medskip

\textbf{Proof:} Let $\epsilon > 0$ be given. Then there exists $\delta > 0$ s.t.\ for all $x,y\in[0,\infty]$ with $|x-y|<\delta$ then $|f(x)-f(y)| < \epsilon$. Choose $\delta = \epsilon$ then

\[f(x)-f(y)| = \left|\frac{x}{x+1}-\frac{y}{y+1}\right|\]

\[= \left|\frac{x-y+xy-xy}{(x+1)(y+1)}\right|\]

\[< \delta\left|\frac{1}{(x+1)(y+1)}\right|\]

Now we will bound $x$ and $y$ from our knowledge of the domain. We know that $\frac{1}{(x+1)(y+1)}$ is a decreasing function so it achieves a maximum at it's smallest value, $0$. So

\[<\delta\left|\frac{1}{(0+1)(0+1)}\right|\]

\[=\delta\left|\frac{1}{1}\right| = \delta = \epsilon\]

Therefore $h(x) = \frac{x}{x+1}$ is uniformly continuous on the interval $[0, \infty)$.

\qedsym

\section*{6}

\textbf{Claim:} $h(x) = \frac{x}{x^2+1}$ is uniformly continuous on the interval $(-\infty, \infty)$

\medskip

\textbf{Proof:} Let $\epsilon > 0$ be given. Then there exists $\delta > 0$ s.t.\ for all $x,y\in\R$ with $|x-y|<\delta$ then $|f(x)-f(y)| < \epsilon$. Choose $\delta = \min\{1, \epsilon\}$ then

\[|f(x) - f(y)| = \left|\frac{1}{x^2+1}-\frac{1}{y^2 + 1}\right|\]

\[= \left|\frac{(x-y)(x+y)+1-1}{(x^2+1)(y^2+1)}\right|\]

\[= \delta\left|\frac{(x+y)}{(x^2+1)(y^2+1)}\right|\]

then we can leverage the triangle inequality

\[< \delta\left(\frac{|x|}{|x^2+1||y^2+1|}+\frac{|y|}{|x^2+1||y^2+1|}\right)\]

{\addtolength{\leftskip}{5mm}
\textit{Aside:}

Let $\ds f(x) = \frac{x}{x^2+1} \Rightarrow f'(x) = \frac{1-x^2}{{(x^2+1)}^2}\Rightarrow$ critical points $x=1,-1$

Let $\ds g(y) = \frac{1}{y^2+1} \Rightarrow g'(y) = \frac{-2y}{{(y^2+1)}^2}\Rightarrow$ critical points $y=0$

Let $\ds i(x) = \frac{y}{y^2+1} \Rightarrow i'(y) = \frac{1-y^2}{{(y^2+1)}^2}\Rightarrow$ critical points $y=1,-1$

Let $\ds j(x) = \frac{1}{x^2+1} \Rightarrow j'(x) = \frac{-2x}{{(x^2+1)}^2}\Rightarrow$ critical points $x=0$

}\medskip

We see from the aside that $f$ is maximized when $x=1, y=0$ so
\[|x-y| = |1-0| = 1 = \delta\]
and $g$ is maximized when $x=0,y=1$ so
\[|x-y| = |0-1| = 1 = \delta\]
the same is true for $i$ and $j$ respectively. So

\[\delta\left(\frac{|x|}{|x^2+1||y^2+1|}+\frac{|y|}{|x^2+1||y^2+1|}\right) < \delta\left(\frac{|1|}{|2||1|}+\frac{|1|}{|1||2|}\right)\]

\[=\delta\left(\frac{1}{2} + \frac{1}{2}\right) = \delta = \epsilon\]

Therefore $h(x) = \frac{x}{x^2+1}$ is uniformly continuous on the interval $(-\infty, \infty)$

\qedsym

\bigskip

For problems 7 and 8 use the sequential characterization of uniform continuity to show that the function is not uniformly continuous.

\section*{7}

\textbf{Claim:} $f(x) = x^2+2x-3$ is not uniformly continuous on the interval $[0, \infty)$

\medskip

\textbf{Proof:}

\section*{8}

\textbf{Claim:} $g(x) = \frac{1}{x+1}$ is not uniformly continuous on the interval $(-1, 1)$

\medskip

\textbf{Proof:} If $g(x)$ is uniformly continuous, then given any two sequences $a_n, b_n$ s.t. \[\ds\lim_{n\to\infty}a_n-b_n = 0\] then \[\ds\lim_{n\to\infty}g(a_n)-g(b_n) = 0\]

Observe the following two sequences
\[a_n = \frac{-n+1}{n} \text{ and } b_n = \frac{-n+3}{n}\]

We can see that for any constant $c$,
\[\lim_{n\to\infty}\frac{-n+c}{n} = \lim_{n\to\infty}-\frac{n}{n} + \frac{c}{n}=\lim_{n\to\infty}-1 + \frac{c}{n}\]

By previous proofs we know that $\ds\lim_{n\to\infty}\frac{c}{n} = 0$ and the limit of a constant function is the constant, so we can use the sum property of limits to say

\[\lim_{n\to\infty}-1 + \frac{c}{n} = \lim_{n\to\infty}-1 + \lim_{n\to\infty}\frac{c}{n}=-1+0=-1\]

So $\lim_{n\to\infty}a_n = -1$ and $\lim_{n\to\infty}b_n=-1$ so we can use the sum property of limits to show that $a_n$ and $b_n$ satisfy the first condition

\[\ds\lim_{n\to\infty}a_n-b_n = \lim_{n\to\infty}a_n - \lim_{n\to\infty}b_n = -1-(-1) = 0\]

Perfect! Now let us consider the second condition.

\[\ds\lim_{n\to\infty}g(a_n)-g(b_n)\]

\[=\ds\lim_{n\to\infty}\frac{1}{a_n+1}-\frac{1}{b_n+1}\]

\[=\ds\lim_{n\to\infty}\frac{1}{\frac{-n+1}{n}+1}-\frac{1}{\frac{-n+3}{n}+1}\]

\[=\ds\lim_{n\to\infty}\frac{1}{-1+\frac{1}{n}+1}-\frac{1}{-1 + \frac{3}{n}+1}\]

\[=\ds\lim_{n\to\infty}\frac{1}{\frac{1}{n}}-\frac{1}{\frac{3}{n}}\]

\[=\lim_{n\to\infty}n-\frac{n}{3}\]

\[=\lim_{n\to\infty}\frac{2}{3}n\]

{\addtolength{\leftskip}{5mm}
\textit{Aside:}

A set $s$ is bounded if there exists $M$ s.t.\ for all $x\in s$, $x\leq M$

For $\ds n>\frac{3M}{2} \Rightarrow \frac{2}{3}n > M$ therefore $\frac{2}{3}n$ is not bounded.

}

From the aside we know that $\ds\lim_{n\to\infty}\frac{2}{3}n$ does not exist, therefore \[\ds\lim_{n\to\infty}g(a_n)-g(b_n) \neq 0\] so $g(x)$ is not uniformly continuous.

\qedsym

\end{document}
