\documentclass[10pt,letterpaper]{article}

\usepackage{amsmath}
\usepackage{amsthm}
\usepackage{amssymb}
\usepackage{amsfonts}
\usepackage{array}

\newcommand\Q{\mathbf{Q}}
\newcommand\Z{\mathbf{Z}}
\newcommand\R{\mathbf{R}}
\newcommand\C{\mathbf{C}}
\newcommand\A{\mathbf{A}}
\newcommand\ds{\displaystyle}

\newcommand{\imp}{\Rightarrow}
\newcommand\qedsym{\hfill \rule{2mm}{2mm}}

\parindent=0pt
\begin{document}

\title{MATH5210 \textsc{Analysis}
  \\ Assignment 4
  \\ Monotone Convergence Theorem
  \\ Philip Nelson
}

\date{}

\maketitle

\section*{1} Let $\{a_n\}$ be a sequence of non-negative real numbers. Show that the partial sums for the infinite series $\ds \sum_{n=1}^{\infty } a_n$ form a monotone series.

\textbf{Proof:} Let us review the definition of a monotone increasing series. Let $b_m$ be a series, $b_m$ is monotone increasing iff $b_m \leq b_{m+1}$ for all $m > 0$. Let $b_m$ be the partial sums of $\ds \sum_{n=1}^{\infty } a_n$ such that $b_1 = a_1$, $b_2 = b_1 + a_2$, and $b_{m+1} = b_m + a_n$. From the definition of the problem, $\{a_n\}$ is a sequence of non-negative real numbers so $a_n \geq 0 \Rightarrow b_m + a_n \geq b_m \Rightarrow b_{m+1} \geq b_m$. Therefore $b_m$ is a monotone series. \qedsym{}
\section*{2} What is the approximation property for the infimum of a set $\A$? If $m = \inf(\A)$, prove that there is a sequence $a_n \in \A$ with $\lim a_n = m$.

\medskip

\textbf{Property:} The approximation property for the infimum of a set $\A$.

Let $\A$ be a set such that $m = \inf(\A)$.

Then for any $\epsilon > 0, \exists x \in \A \text{ s.t. }  m \leq x < m + \epsilon$

\medskip

\textbf{Claim:} If $m = \inf(\A)$, there is a sequence $a_n \in \A$ with $\lim a_n = m$.

\medskip

\textbf{Proof:} Let $\epsilon_1 = 1$ and $\epsilon_2 = \frac{1}{2}$. By the approximation property, there exist $x_1$ and $x_2$ such that $m \leq x_1 < m + 1$ and $m \leq x_2 < m + \frac{1}{2}$. Then for any $n\ \exists x_n \in \A$ s.t. $m \leq x_n < m + \frac{1}{n}$. Let $a_n = x_n$. Therefore there is a sequence $a_n$ that we can prove has a limit which converges to $m$. We will make the proof by the squeeze theorem that $\lim a_n = m$.

We know that $m \leq a_n < m + \frac{1}{n}$ so we will take the limit of each part of the inequality so $\ds\lim_{n\to\infty}m \leq \ds\lim_{n\to\infty}x_n < \ds\lim_{n\to\infty}m + \frac{1}{n}$. We will look at the limits and see that $\ds\lim_{n\to\infty}m = m$ because $m$ is a constant sequence and $\ds\lim_{n\to\infty}m+\frac{1}{n} = m$ because we know that $\ds\lim_{n\to\infty}\frac{1}{n} = 0$ from previous $\epsilon - N$ proofs and therefore by the summation property of limits $\ds\lim_{n\to\infty}m + \ds\lim_{n\to\infty}\frac{1}{n} = 0 + m = m$. So we can see that $m \leq \ds\lim_{n\to\infty}a_n \leq m$ so by the squeeze theorem of limits $\ds\lim_{n\to\infty} a_n = m$ \qedsym{}

\section*{3} Give a detailed proof of the MCT in the case of a monotone decreasing sequence which is bounded below. Follow the proof given in class.

\medskip

\textbf{Completeness Axiom:} Let $\{a_n\}$ be a set that is bounded below. Then $\{a_n\}$ has an infimum s.t. $\inf(a_n) = A$.

\medskip

\textbf{Claim:} $\ds\lim_{n\to\infty}a_n = A$

\medskip

\textbf{Proof:} Let $\epsilon > 0$ be given, find $N$ s.t.

\begin{equation}
  |a_n - A| < \epsilon
\end{equation}

The set is bounded below so the quantity $a_n - A$ is positive. We can express equation 1 as

\begin{equation}
  a_n - A < \epsilon
\end{equation}

By the approximation property of the infimum, there is an $x$ in $\{a_n\}$ s.t. $A \leq x < A + \epsilon$. Let $x = a_N$ then

\begin{equation}
  a_N < A + \epsilon
\end{equation}

If $n > N$, and since $\{a_n\}$ is monotonically decreasing

\begin{equation}
  a_n > a_N
\end{equation}

Combining equations 3 and 4 we get \[a_n < a_N < A + \epsilon\] this implies \[a_n - A < \epsilon\] as in equation 2. \qedsym{}

\section*{4} If $0 < r < 1$, show that $\ds \lim_{n \to \infty} r^n= 0$. Note that with $a_n = r^n$, $a_{n+1} = r a_n$. What happens if $0 \leq r \leq 1$?

\medskip

\textbf{Lemma 3.1:} $r^n > 0$ for all $n > 0$ and $0 < r < 1$.

Let us observe $a_1 = r^1$ is positive because $0 < r < 1$ and $a_2 = r^2 = r\cdot r$ is positive because a positive number times a positive number is positive. Assume $a_n > 0$, then $a_{n+1} = r\cdot a_n$ which is positive because as previously stated, a positive number times a positive number is positive. Therefore $r^n > 0$ for all $n > 0$ and $0 < r < 1$.

\medskip

\textbf{Lemma 3.2:} $a_n$ is monotone decreasing.

From the bounds of $r$ we know that $r<1$. By lemma 3.1 we know that $r^n > 0$ so we can multiply both sides by $r^n$ while preserving inequality so $r\cdot r^n < r^n$. Then we can use the definition $a_n = r^n$ so we have $r\cdot a_n < a_n$. Then we can use the definition $a_{n+1} = r\cdot r^n$ so we have $a_{n+1} < a_n$. Therefore the sequence is monotone decreasing.

\medskip

\textbf{Claim:} $\ds \lim_{n \to \infty} r^n= 0$ for $0 < r < 1$.

\medskip

\textbf{Proof:} Through lemma 3.1 and 3.2, we know that $a_n$ is bounded below by zero and monotone decreasing so by the Monotone Convergence Theorem we know that the limit of $a_n$ exists. Let $A = \ds \lim_{n \to \infty} r^n$. Then starting with $a_{n+1} = r\cdot a_n$ we take the limit of both sides giving us $\ds \lim_{n \to \infty} a_{n+1} = \ds \lim_{n\to\infty} r \cdot r^n$. By the subsequence property of limits we have $A = \ds \lim_{n\to\infty} r \cdot r^n$. Then by the scalar product property of limits, we have $A = r\cdot A$ which is $r\cdot A - A = 0$ and so $A (r - 1) = 0$. This can be satisfied by only $A = 0$ therefore $\ds \lim_{n \to \infty} r^n= 0$. \qedsym{

  \section*{5} Let $a_1 \in (-1, 0)$ Prove that $a_{n+1} = \sqrt{a_{n} +1} - 1$ is a increasing sequence which converges to 0. What happens if $a_0 = 0$, if $a_0 = -1$?

\medskip

\textbf{Lemma 5.1} $a_{n+1} = \sqrt{a_{n} +1} - 1 < 0$ for $a_1 \in (-1, 0)$

From the problem statement we know that $-1 < a_n < 0$. Then if $-1 < a_n < 0$ then $0 < a_n + 1 < 1$. Since the square root is an increasing function, we can apply it whilst preserving inequality so $0 < \sqrt{a_n + 1} < 1$ then $-1 < \sqrt{a_n + 1} - 1< 0$ which is $-1 < a_{n+1} < 0$. Therefore $a_{n+1} = \sqrt{a_{n} +1} - 1 < 0$ for $a_1 \in (-1, 0)$.

\medskip

\textbf{Lemma 5.2} $a_{n+1} = \sqrt{a_{n} +1} - 1$ is monotone increasing for $a_1 \in (-1, 0)$.

If $a_1 = -0.5$ then $a_{n+1} = \sqrt{-0.5 + 1} - 1 \approx -0.29$ therefore $a_1 < a_2$. If $a_n < a_{n+1}$ then $a_n + 1 < a_{n+1} + 1$. Since the square root function is an increasing function, we can apply it whilst preserving inequality so $\sqrt{a_n + 1} < \sqrt{a_{n+1} + 1}$ then $\sqrt{a_n + 1} - 1 < \sqrt{a_{n+1} + 1} - 1$ which is $a_{n+1} < a_{n+2}$. Therefore $a_{n+1} = \sqrt{a_{n} +1} - 1$ is monotone increasing for $a_1 \in (-1, 0)$.

\medskip

\textbf{Claim:} $\ds\lim_{n\to\infty} a_{n} = 0$ for $a_1 \in (-1. 0)$

\medskip

\textbf{Proof:} Through lemma 5.1 and 5.2 we know that $a_n$ is bounded above by $0$ and is an increasing sequence. Therefore we know by the Monotone Convergence Theorem that $\ds\lim_{n\to\infty} a_n$ exists. Let $A = \ds\lim_{n\to\infty}a_n$. Then starting with the definition $a_{n+1} = \sqrt{a_{n} +1} - 1$ we take the limit of both sides so $\ds\lim_{n\to\infty}a_{n+1} = \ds\lim_{n\to\infty}\sqrt{a_{n} +1} - 1$. Let $b_n = 1$ and remark that $\sqrt{a_n + 1} -1 = \sqrt{a_n + b_n} - b_n$. Let $c_n = \sqrt{a_n - b_n}$. Now see that $\ds\lim_{n\to\infty} b_n = \ds\lim_{n\to\infty} 1 = 1$ since $b_n$ is a constant sequence and $\ds\lim_{n\to\infty} a_n = A$ as previously stated. Then because the limit exists we see that $\ds\lim_{n\to\infty} a_n + 1 = A + 1$ by the summation property. So by the summation property and square root property we can express the original limit as $\ds\lim_{n\to\infty}a_{n+1} = \sqrt{\ds\lim_{n\to\infty}a_{n} +\ds\lim_{n\to\infty}1} - \ds\lim_{n\to\infty}1$. Then by the subsequence property and evaluation of limits as previously discussed $A = \sqrt{A+1}-1$. Then $A+1 = \sqrt{A+1}$ and then $A^2 + 2A + 1 = A + 1$ and then $A^2 + A = 0$ so $A = -1, 0$. Returning to the bounds set on $a_1$ we know that $-1 < a_1 0$ and since $a_n$ is an increasing sequence $A \neq -1$. Therefore $\ds\lim_{n\to\infty} a_{n} = 0$ for $a_1 \in (-1. 0)$. \qedsym{}

\medskip

\textbf{Claim:} $\ds\lim_{n\to\infty} a_{n} = 0$ for $a_1 = 0$

\medskip

\textbf{Proof:} If $a_1 = 0$ then $a_2 = \sqrt{1} - 1 = 0$, therefore $a_1 = a_2$. If $a_n = a_{n+1}$ then $a_n + 1= a_{n+1} +1$ Since the square root is an increasing function we can apply whilst preserving inequality so $\sqrt{a_n + 1} = \sqrt{a_{n+1} +1}$ then $\sqrt{a_n + 1} - 1 = \sqrt{a_{n+1} +1} - 1$ which is $a_{n+1} = a_{n+2}$. Therefore $\ds\lim_{n\to\infty} a_{n} = 0$ for $a_1 = 0$. \qedsym{}

\medskip

\textbf{Claim:} $\ds\lim_{n\to\infty} a_{n} = 1$ for $a_1 = 1$

\medskip

\textbf{Proof:} If $a_1 = -1$ then $a_2 = \sqrt{0} - 1 = -1$, therefore $a_1 = a_2$. If $a_n = a_{n+1}$ then $a_n + 1= a_{n+1} +1$ Since the square root is an increasing function we can apply whilst preserving inequality so $\sqrt{a_n + 1} = \sqrt{a_{n+1} +1}$ then $\sqrt{a_n + 1} - 1 = \sqrt{a_{n+1} +1} - 1$ which is $a_{n+1} = a_{n+2}$. Therefore $\ds\lim_{n\to\infty} a_{n} = -1$ for $a_1 = -1$. \qedsym{}

\section*{6} Suppose $a_1 \geq 2$ and $a_{n +1} = 2 + \sqrt{a_n - 2}$. Show that $a_n$ converges to 2 or 3. How does the limit depend upon the value of $a_1$

\medskip

\textbf{Lemma 6.1:} $a_n$ is bounded above by $3$ for all $n > 0$ and $2 < a_n < 3$.

Assuming $2 < a_n < 3$ then $0 < a_n - 2 < 1$. Since the square root function is an increasing function, we can apply it whilst preserving inequality so $0 < \sqrt{a_n - 2} < 1$ then $2 < \sqrt{a_n - 2} + 2 < 3$. Since
$a_{n +1} = 2 + \sqrt{a_n - 2}$ we can see $2 < a_{n+1} < 3$. Therefore $a_n$ is bounded above by $3$ for all $n > 0$ and $2 < a_n < 3$.

\medskip

\textbf{Lemma 6.2:} $a_{n +1} = 2 + \sqrt{a_n - 2}$ is monotone increasing for $2 < a_n < 3$.

Observe $a_1 = 2.1$ then $a_2 \approx 2.3$ therefore $a_1 < a_2$. If $a_n < a_{n+1}$ then $a_n - 2 < a_{n+1} - 2$. Since the square root is a positive function, we can apply it to both sides and maintain inequality. So $\sqrt{a_n - 2} < \sqrt{a_{n+1} - 2}$. Then $2 + \sqrt{a_n - 2} < 2 + \sqrt{a_{n+1} - 2}$ which means that $a_{n+1} < a_{n+2}$. Therefore $a_{n +1} = 2 + \sqrt{a_n - 2}$ is monotone increasing for $2 < a_n < 3$.

\medskip

\textbf{Lemma 6.3:} $\ds \lim_{n\to\infty}a_n = 3$ for $2 < a_n < 3$

By lemma 6.1 and 6.2 we know that $a_n$ is bounded above by $3$ and is monotone increasing for $2 < a_n < 3$ so by the Monotone Convergence Theorem we know $\ds \lim_{n\to\infty}a_n$ exists on these bounds.
Let $A = \ds \lim_{n\to\infty} a_n$.
Starting with $a_{n+1} = 2 + \sqrt{a_n - 2}$ we take the limit of both sides so $\ds\lim_{n\to\infty}a_{n+1} = \ds\lim_{n\to\infty} 2 + \sqrt{a_n - 2}$.
Let $b_n = 2$ and remark that $2 + \sqrt{a_n - 2} = b_n + \sqrt{a_n - b_n}$.
Let $c_n = \sqrt{a_n - b_n}$.
Now see that $\ds\lim_{n\to\infty} b_n = \ds\lim_{n\to\infty} 2 = 2$ since $b_n$ is a constant sequence and $\ds\lim_{n\to\infty} a_n = A$ as previously stated.
Then because the limit exists we see that $\ds\lim_{n\to\infty} a_n - 2 = A - 2$ by the summation property.
So by the summation property and the square root property we can express the original limit as $\ds\lim_{n\to\infty} a_{n+1} = \ds\lim_{n\to\infty} 2 + \sqrt{\ds\lim_{n\to\infty}a_n - \ds\lim_{n\to\infty}2}$.
Then by the subsequence property of limits and evaluation of limits as previously discussed $A = 2 + \sqrt{A - 2}$
Then $A - 2 = \sqrt{A - 2}$ and then $A^2 - 4A + 2 = A - 2$ and then $A^2 - 5A + 6 = 0$ and then $(A - 3)(A - 2) = 0$ so $A = 2,3$ are fixed points. Returning to our bounds on $a_1$ we know that $A \neq 2$ since $2 < a_1 < 3$ and $a_n$ is monotone increasing for $2 < a_n < 3$. Therefore $\ds \lim_{n\to\infty}a_n = 3$ for $2 < a_n < 3$.

\medskip

\textbf{Lemma 6.4} $a_n$ is bounded below by $3$ for all $n > 0$ and $a_n > 3$

Assuming $a_n > 3$ then $a_n - 2 > 1$. Since the square root function is an increasing, we can apply it whilst preserving inequality so $\sqrt{a_n - 2} > 1$. Then it follows that $2 + \sqrt{a_n - 2} > 3$. Therefore $a_n$ is bounded below by $3$ for all $n > 0$ and $a_n > 3$

\medskip

\textbf{Lemma 6.5:} $a_{n +1} = 2 + \sqrt{a_n - 2}$ is monotone decreasing for $a_n > 3$.

Observe $a_1 = 4$ then $a_2 \approx 3.4$ therefore $a_1 > a_2$. If $a_n > a_{n+1}$ then $a_n - 2 > a_{n+1} - 2$. Since the square root is a positive function, we can apply it to both sides and maintain inequality. So $\sqrt{a_n - 2} > \sqrt{a_{n+1} - 2}$. Then $2 + \sqrt{a_n - 2} > 2 + \sqrt{a_{n+1} - 2}$ which means that $a_{n+1} > a_{n+2}$. Therefore $a_{n +1} = 2 + \sqrt{a_n - 2}$ is monotone decreasing for $a_n > 3$.

\medskip

\textbf{Lemma 6.6:} $\ds \lim_{n\to\infty}a_n = 3$ for $a_n > 3$

By lemma 6.4 and 6.5 we know that $a_n$ is bounded below by $3$ and is monotone decreasing for $a_n > 3$ so by the Monotone Convergence Theorem we know $\ds \lim_{n\to\infty}a_n$ exists on these bounds.
Let $B = \ds \lim_{n\to\infty} a_n$.
Starting with $a_{n+1} = 2 + \sqrt{a_n - 2}$ we take the limit of both sides so $\ds\lim_{n\to\infty}a_{n+1} = \ds\lim_{n\to\infty} 2 + \sqrt{a_n - 2}$.
Let $b_n = 2$ and remark that $2 + \sqrt{a_n - 2} = b_n + \sqrt{a_n - b_n}$.
Let $c_n = \sqrt{a_n - b_n}$.
Now see that $\ds\lim_{n\to\infty} b_n = \ds\lim_{n\to\infty} 2 = 2$ since $b_n$ is a constant sequence and $\ds\lim_{n\to\infty} a_n = B$ as previously stated.
Then because the limit exists we see that $\ds\lim_{n\to\infty} a_n - 2 = B - 2$ by the summation property.
So by the summation property and the square root property we can express the original limit as $\ds\lim_{n\to\infty} a_{n+1} = \ds\lim_{n\to\infty} 2 + \sqrt{\ds\lim_{n\to\infty}a_n - \ds\lim_{n\to\infty}2}$.
Then by the subsequence property of limits and evaluation of limits as previously discussed $B = 2 + \sqrt{B - 2}$
Then $B - 2 = \sqrt{B - 2}$ and then $B^2 - 4B + 2 = B - 2$ and then $B^2 - 5B + 6 = 0$ and then $(B - 3)(B - 2) = 0$ so $B = 2,3$ are fixed points. Returning to our bounds on $a_1$ we know that $B \neq 2$ since $a_1 > 3$ and $a_n$ is monotone decreasing for $a_n > 3$. Therefore $\ds \lim_{n\to\infty}a_n = 3$ for $a_n > 3$.

\medskip

\textbf{Lemma 6.7} $\ds \lim_{n\to\infty}a_n = 3$ for $a_1 = 3$

If $a_1 = 3$ then $a_2 = 2+\sqrt{3-2}=3$, therefore $a_1 = a_2$. If $a_n = a_{n+1}$ then $a_n - 2= a_{n+1} - 2$. Since the square root is an increasing function we can apply whilst preserving inequality so $\sqrt{a_n - 2} = \sqrt{a_{n+1} - 2}$ then $2+\sqrt{a_n - 2} = 2+\sqrt{a_{n+1} - 2}$ which is $a_{n+1} = a_{n+2}$. Therefore $\ds \lim_{n\to\infty}a_n = 3$ for $a_1 = 3$.

\medskip

\textbf{Lemma 6.8} $\ds \lim_{n\to\infty}a_n = 2$ for $a_1 = 2$

If $a_1 = 2$ then $a_2 = 2+\sqrt{2-2}=2$, therefore $a_1 = a_2$. If $a_n = a_{n+1}$ then $a_n - 2= a_{n+1} - 2$. Since the square root is an increasing function we can apply whilst preserving inequality so $\sqrt{a_n - 2} = \sqrt{a_{n+1} - 2}$ then $2+\sqrt{a_n - 2} = 2+\sqrt{a_{n+1} - 2}$ which is $a_{n+1} = a_{n+2}$. Therefore $\ds \lim_{n\to\infty}a_n = 3$ for $a_1 = 2$.

\medskip

\textbf{Proof:} For $a_1 \geq 2$ and $a_{n +1} = 2 + \sqrt{a_n - 2}$. $\ds\lim_{n\to\infty}a_n$ converges to 2 or 3.

From lemmas 6.3, 6.6, 6.7 and 6.8 we know that for $a_1 \geq 2$, $\ds\lim_{n\to\infty}a_n$ can only converge to $2$ or $3$. It depends on $a_1$ which value $\ds\lim_{n\to\infty}a_n$ converges to. When $a_1 = 2$, then $\ds\lim_{n\to\infty}a_n = 2$, lemma 6.8. When $a_1 > 2$, then $\ds\lim_{n\to\infty}a_n = 3$. \qedsym{}

% \section*{7} (Bonus) Let $0 < b_1 < a_1$ and let
%
% \[a_{n+1} = \frac{a_n + b_n}{2} \quad \text{and} \quad b_{n+1} = \sqrt{a_n b_n}\]
%
% \subsection*{a} Show that $0 < b_n < a_n$
%
%
% \subsection*{b} Show that $b_n$ is increasing and bounded above.
%
%
% \subsection*{c} Show that $a_n$ is decreasing and bounded below.
%
%
% \subsection*{d} Show that $ 0 < a_{n+1} - b_{n+1} < (a_1 - b_1)/2^n$
%
%
% \subsection*{e} Conclude that the sequences $a_n$ and $b_n$ converge to a common limit.
%
%
\end{document}







