\documentclass[10pt,letterpaper]{article}

\usepackage{amsmath}
\usepackage{amsthm}
\usepackage{amssymb}
\usepackage{amsfonts}
\usepackage{array}

\newcommand\Q{\mathbf{Q}}
\newcommand\Z{\mathbf{Z}}
\newcommand\R{\mathbf{R}}
\newcommand\C{\mathbf{C}}
\newcommand\A{\mathbf{A}}
\newcommand\ds{\displaystyle}

\newcommand{\imp}{\Rightarrow}
\newcommand\qedsym{\hfill \rule{2mm}{2mm}}

\parindent=0pt
\begin{document}

\title{MATH5210 \textsc{Analysis}
  \\ Assignment 4
  \\ Monotone Convergence Theorem
  \\ Philip Nelson
}

\date{}

\maketitle

\section*{1} Let $\{a_n\}$ be a sequence of non-negative real numbers. Show that the partial sums for the infinite series $\ds \sum_{n=1}^{\infty } a_n$ form a monotone series.

\section*{2} What is the approximation property for the infimum of a set $\A$? If $m = \inf(\A)$, prove that there is a sequence $a_n \in \A$ with $\lim a_n = m$.

\bigskip

The approximation property for the infimum of a set $\A$ states

Suppose that $m = \inf(\A)$

Then for any $\epsilon > 0, \exists x \in \A \text{ s.t. }  m \leq x \leq m + \epsilon$

\bigskip

Since $m = \inf(\A)$, $\{a_n\}$ is bounded below and has a greatest upper bound (gub) s.t. $\text{gub}(\{a_n\}) = m$.
We will prove will prove that $\lim\limits_{n \to \infty} a_n = m$.

Let $\epsilon > 0$ be given. We have to find $N$ s.t. $| a_n - m | < \epsilon$ and $a_n - m$ is always positive because $a_n > m$. So $a_n - m < \epsilon$ for all $n > N$.

Then by the approximation property of the infimum of $\A$, there exists an $x \in \{a_n\}$ s.t. $m \leq x < m + \epsilon$ where $x = a_N$. So

\begin{equation}
a_N \leq m + \epsilon
\end{equation}

Finally if $n>N$, then because of monotone decreasing

\begin{equation}
a_N < a_n
\end{equation}

Combining 1 and 2 gives us $m + \epsilon > a_N > a_n$.
Thus $a_n - m < \epsilon$ for all $n > N$.

\qedsym{}

\section*{3} Give a detailed proof of the MCT in the case of a monotone decreasing sequence which is bounded below. Follow the proof given in class.

\section*{4} If $0 < r < 1$, show that $\ds \lim_{n \to \infty} r^n= 0$. Note that with $a_n = r^n$, $a_{n+1} = r a_n$. What happens if $0 \leq r \leq 1$?

\section*{5} Let $a_1 \in (0, 1)$ Prove that $a_{n+1} = \sqrt{a_{n} +1} -1 $ is a increasing sequence which converges to 0. What happens if $a_0 = 0$, if $a_0 = -1$?

\section*{6} Suppose $a_1 \geq 2$ and $a_{n +1} = 2 + \sqrt{a_n - 2}$. Show that $a_n$ converges to 2 or 3. How does the limit depend upon the value of $a_1$

\section*{7} (Bonus) Let $0 < b_1 < a_1$ and let

\[a_{n+1} = \frac{a_n + b_n}{2} \quad \text{and} \quad b_{n+1} = \sqrt{a_n b_n}\]

\subsection*{a} Show that $0 < b_n < a_n$


\subsection*{b} Show that $b_n$ is increasing and bounded above.


\subsection*{c} Show that $a_n$ is decreasing and bounded below.


\subsection*{d} Show that $ 0 < a_{n+1} - b_{n+1} < (a_1 - b_1)/2^n$


\subsection*{e} Conclude that the sequences $a_n$ and $b_n$ converge to a common limit.


\end{document}







