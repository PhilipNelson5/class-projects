\documentclass[10pt,letterpaper]{article}

\usepackage{amsmath}
\usepackage{amsthm}
\usepackage{amssymb}
\usepackage{amsfonts}
\usepackage{array}

\newcommand\Q{\mathbf{Q}}
\newcommand\Z{\mathbf{Z}}
\newcommand\R{\mathbf{R}}
\newcommand\C{\mathbf{C}}
\newcommand\A{\mathbf{A}}
\newcommand\ds{\displaystyle}

\newcommand{\imp}{\Rightarrow}
\newcommand\qedsym{\hfill \rule{2mm}{2mm}}

\parindent=0pt
\begin{document}

\title{MATH5210 \textsc{Analysis}
  \\ Assignment 6
  \\ Limits of Functions pt. 2
  \\ Philip Nelson
}

\date{}

\maketitle

\section*{3}

\textbf{Claim:} Let $f, g, h:(a, b)\rightarrow\R$ s.t. $f(x) \leq g(x) \leq h(x)$ for all $x_0\in(a, b)$. If $\ds\lim_{x\to x_0} f(x) = L$ and $\ds\lim_{x\to x_0} h(x) = M$ and $L=M$, then $\ds\lim_{x\to x_0} h(x)$ exists and equals $L$.

\medskip

\textbf{Proof:} Let $\epsilon > 0$ be given. I will show that $\exists\ \delta > 0$ s.t. $|g(x) - L| < \epsilon$ for $0 < |x-x_0| < \delta$.

We know $\exists\ \delta_1$ s.t.

\[|f(x)-L|<\epsilon\]

\[\Rightarrow -\epsilon<f(x)-L<\epsilon\]

\[\Rightarrow L-\epsilon<f(x)<L+\epsilon\text{, for } 0<|x-x_0|<\delta_1\]

\medskip
We also know $\exists\ \delta_2$ s.t.

\[|h(x)-M|<\epsilon\]

\[\Rightarrow -\epsilon<h(x)-M<\epsilon\]

\[\Rightarrow M-\epsilon<h(x)<M+\epsilon\text{, for } 0<|x-x_0|<\delta_2\]

\medskip
Let $\delta = \min\{\delta_1, \delta_2\}$, then for any $x$ s.t. $0 < |x-x_0| < \delta$

\[L-\epsilon < f(x) \leq g(x) \leq h(x) < M+\epsilon\] 

and since $L=M$

\[\Rightarrow L-\epsilon < g(x) < L+\epsilon\] 

\[\Rightarrow |g(x) - L| < \epsilon\text{, for }0<|x-x_0|<\delta\]
Therefore, if $\ds\lim_{x\to x_0} f(x) = L$ and $\ds\lim_{x\to x_0} h(x) = M$ and $L=M$, then $\ds\lim_{x\to x_0} h(x)$ exists and equals $L$.

\qedsym

\section*{5}

\textbf{Claim:} Let $f,g:(a, b)\rightarrow\R$ and let $x_0\in(a, b)$. If $\ds\lim_{x\to x_0} f(x) = L$ and $\ds\lim_{x\to x_0} g(x) = M$, then $\ds\lim_{x\to x_0} f(x)g(x) = LM$.

\medskip

\textbf{Proof:} I will use sequential characterization of limits to prove the claim. For \textbf{all} sequences $a_n$ s.t.

\[\ds\lim_{n\to\infty} a_n = x_0\]

then 

\[\ds\lim_{n\to\infty}f(a_n) = L\]

and

\[\ds\lim_{n\to\infty}g(a_n) = M\]

Let $b_n = f(a_b)$ and $c_n = g(a_n)$ then

\[\ds\lim_{n\to\infty}b_n = \ds\lim_{n\to\infty}f(a_n) = L\]

and

\[\ds\lim_{n\to\infty}b_n = \ds\lim_{n\to\infty}f(a_n) = L\]

Observe $\ds\lim_{n\to\infty}b_nc_n$. We know $\ds\lim_{n\to\infty}b_n$ exists and $\ds\lim_{n\to\infty}c_n$ exists, so by the product property of sequences

\[\ds\lim_{n\to\infty}b_nc_n = \ds\lim_{n\to\infty}b_n \cdot \ds\lim_{n\to\infty}c_n = LM\]

Therefore if $\ds\lim_{x\to x_0} f(x) = L$ and $\ds\lim_{x\to x_0} g(x) = M$, then $\ds\lim_{x\to x_0} f(x)g(x) = LM$.

\qedsym

\section*{6}


\textbf{Claim:} If $\ds\lim_{x\to x_0} f(x) = L$ and $\ds\lim_{y\to L} g(y) = M$, then $\ds\lim_{x\to x_0} g \circ f(x) = M$

\medskip

\textbf{Proof:}

\textit{Remark}: For all sequences $a_n$ s.t. $\ds\lim_{n\to\infty} a_n = x_0$ then 
\[\lim_{n\to\infty} f(a_n) = L\]
Let $b_n = f(a_n)$

\textit{Remark}: For all sequences $c_n$ s.t. $\ds\lim_{n\to\infty} c_n = L$ then 
\[\lim_{n\to\infty} g(c_n) = M\]
Let $b_n = f(a_n)$

Note that $b_n$ is a sequence s.t. $\ds\lim_{n\to\infty}b_n=L$. $b_n$ satisfies the sequential characterization of $g(x)$ therefore
\[\lim_{n\to\infty} g(b_n) = M\]

Since we defined $b_n = f(a_n)$ then
\[\lim_{n\to\infty} g(b_n) = \lim_{n\to\infty} g(f(a_n))\]
\[= \lim_{n\to\infty} g(L) = M\]

So, since $\lim_{n\to\infty} g(f(a_n)) = M$ then
\[\lim_{n\to\infty} g(f(a_n)) = \lim_{n\to\infty} g\circ f(x) = M\]

Therefore if $\ds\lim_{x\to x_0} f(x) = L$ and $\ds\lim_{y\to L} g(y) = M$, then $\ds\lim_{x\to x_0} g \circ f(x) = M$

\qedsym

\section*{9 $\epsilon - \delta$}

\textbf{Claim:} $\ds\lim_{x\to 1}\frac{x^2+4}{x^2-4} = -\frac{5}{3}$

\medskip

\textbf{Proof:} For all $\epsilon > 0$ there exists $\delta = \min\{1, \frac{\epsilon}{2}\}$ s.t. for all $x$, where $|x-1|<\delta$, $|f(x) - L| < \epsilon$. Then

\[|f(x) - L| = \left|\frac{x^2+4}{x^2-4}+\frac{5}{3}\right|\]

\[=\left|\frac{8x^2-8}{3(x^2-4)}\right|\]

\[=\frac{8}{3}\left|\frac{(x+1)(x-1)}{(x+2)(x-2)}\right|\]

\[<\frac{8}{3}\delta\left|\frac{(x+1)}{(x+2)(x-2)}\right|\]


\section*{10 $\epsilon - \delta$}

\textbf{Claim:} $\ds\lim_{x\to 0} \sqrt{x^2 + 1} = 1$

\medskip

\textbf{Proof:}

\section*{11 $\epsilon - \delta$}

\textbf{Claim:} $\ds\lim_{x\to -1} x^3 + x + 2 = 0$

\medskip

\textbf{Proof:}

\section*{12 - 7 limit properties}

\textbf{Claim:} $\ds\lim_{x\to 2} x^2 + x - 5 = 1$

\medskip

\textbf{Proof:}

\section*{12 - 8 limit properties}

\textbf{Claim:} $\ds\lim_{x\to 1} \frac{x}{x^2 + 4} = \frac{1}{5}$

\medskip

\textbf{Proof:}

\section*{13 - 9 limit properties}

\textbf{Claim:} $\ds\lim_{x\to 1}\frac{x^2+4}{x^2-4} = -\frac{5}{3}$

\medskip

\textbf{Proof:}

\section*{13 - 10 limit properties}

\textbf{Claim:} $\ds\lim_{x\to 0} \sqrt{x^2 + 1} = 1$

\medskip

\textbf{Proof:}

\end{document}
