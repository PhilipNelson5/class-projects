\documentclass[10pt,letterpaper]{article}

\usepackage{amsmath}
\usepackage{amsthm}
\usepackage{amssymb}
\usepackage{amsfonts}
\usepackage{array}

\newcommand\Q{\mathbf{Q}}
\newcommand\Z{\mathbf{Z}}
\newcommand\R{\mathbf{R}}
\newcommand\C{\mathbf{C}}
\newcommand\A{\mathbf{A}}
\newcommand\ds{\displaystyle}

\newcommand{\imp}{\Rightarrow}
\newcommand\qedsym{\hfill \rule{2mm}{2mm}}

\parindent=0pt
\begin{document}

\title{MATH5210 \textsc{Analysis}
  \\ Assignment 5
  \\ Limits of Functions
  \\ Philip Nelson
}

\date{}

\maketitle

\section*{1} 

\textbf{Lemma 1.1:} If $\alpha$ is a number s.t. $0\leq\alpha<\epsilon$, then $\alpha = 0$

\medskip

\textbf{Proof:} Assume that $\alpha > 0$, let $\epsilon = \frac{\alpha}{2}$.

Then \[\alpha < \frac{\alpha}{2}\] This conclusion is not possible, then the assumption must be incorrect. Therefore $\alpha = 0$.

\medskip

\textbf{Claim:} Let $f:(a,b)\to\R$ and let $x_0\in(a,b)$. If $\ds\lim_{x\to x_0} f(x)$ exists, then the limit is unique.

\medskip

\textbf{Proof:} Let $\epsilon > 0$ be given.

Then $\exists\ \delta_1$ s.t. \[|f(x) - L| < \frac{\epsilon}{2}\] for all $x$ s.t. \[0 < |x-x_0| < \delta_1\]

Likewise $\exists\ \delta_2$ s.t. \[|f(x) - M| < \frac{\epsilon}{2}\] for all $x$ s.t. \[0 < |x-x_0| < \delta_2\]

Then for $\delta = \min\{\delta_1, \delta_2\}$,

\[|L-M| = |L-f(x) + f(x) - B|\]

and then by the triangle inequality

\[|L-f(x) + f(x) - B| < |L - f(x)| + |f(x) - B|\]

which is equal to

\[=|\frac{\epsilon}{2}|+|\frac{\epsilon}{2}|\]

Since $\epsilon > 0$ was given

\[=\frac{\epsilon}{2}+\frac{\epsilon}{2} = \epsilon\]

So $0 \leq A-B < \epsilon$ and by lemma 1.1,

\[A-B = 0\]

\[\Rightarrow A=B\]

Therefore if $\ds\lim_{x\to x_0} f(x)$ exists, then the limit is unique. \qedsym

\section*{2}

\textbf{Claim:} Let $f:(a, b)\rightarrow\R$ and let $x_0\in(a, b)$. If $\ds\lim_{x\to x_0} f(x) = L$ and $L>0$, then $\exists\ \alpha, \delta>0$ s.t. $f(x)>\alpha\ \forall\ x\in(x_0-\delta, x_0+\delta)$.

\medskip

\textbf{Proof:} Since the limit exists, this is true for $\epsilon = \frac{L}{3},\ \exists\ \delta_1$ s.t. $0<|x-x_0| < \delta_1$.

Using the definition of convergence,
\[|f(x) - L| < \frac{L}{3}\]

\[\Rightarrow -\frac{L}{3} < f(x) - L < \frac{L}{3}\]

\[\Rightarrow L-\frac{L}{3} < f(x) < L+\frac{L}{3}\]

\[\Rightarrow \frac{2}{3}L < f(x) < \frac{4}{3}L\]

So let $\alpha = \frac{2}{3}L$, then for all $x$ s.t. $0 < |x-x_0| < \delta_1$

\[0 < \alpha = \frac{2}{3}L < f(x)\]

Therefore, if $\ds\lim_{x\to x_0} f(x) = L$ and $L>0$, then $\exists\ \alpha, \delta>0$ s.t. $f(x)>\alpha\ \forall\ x\in(x_0-\delta, x_0+\delta)$.

\qedsym

\section*{3}

\textbf{Claim:} Let $f, g, h:(a, b)\rightarrow\R$ s.t. $f(x) \leq g(x) \leq h(x)$ for all $x_0\in(a, b)$. If $\ds\lim_{x\to x_0} f(x) = L$ and $\ds\lim_{x\to x_0} h(x) = M$ and $L=M$, then $\ds\lim_{x\to x_0} h(x)$ exists and equals $L$.

\medskip

\textbf{Proof:} Let $\epsilon > 0$ be given. I will show that $\exists\ \delta > 0$ s.t. $|g(x) - L| < \epsilon$ for $0 < |x-x_0| < \delta$.

We know $\exists\ \delta_1$ s.t.

\[|f(x)-L|<\epsilon\]

\[\Rightarrow -\epsilon<f(x)-L<\epsilon\]

\[\Rightarrow L-\epsilon<f(x)<L+\epsilon\text{, for } 0<|x-x_0|<\delta_1\]

\medskip
We also know $\exists\ \delta_2$ s.t.

\[|h(x)-M|<\epsilon\]

\[\Rightarrow -\epsilon<h(x)-M<\epsilon\]

\[\Rightarrow M-\epsilon<h(x)<M+\epsilon\text{, for } 0<|x-x_0|<\delta_2\]

\medskip
Let $\delta = \min\{\delta_1, \delta_2\}$, then for any $x$ s.t. $0 < |x-x_0| < \delta$

\[L-\epsilon < f(x) \leq g(x) \leq h(x) < M+\epsilon\] 

and since $L=M$

\[\Rightarrow L-\epsilon < g(x) < L+\epsilon\] 

\[\Rightarrow |g(x) - L| < \epsilon\text{, for }0<|x-x_0|<\delta\]
Therefore, if $\ds\lim_{x\to x_0} f(x) = L$ and $\ds\lim_{x\to x_0} h(x) = M$ and $L=M$, then $\ds\lim_{x\to x_0} h(x)$ exists and equals $L$.

\qedsym

\section*{4}

\textbf{Claim:} Let $f:(a, b)\rightarrow\R$ and let $x_0\in(a, b)$. If $f(x) \geq 0$ for all $x\in(a,b)$ and $\ds\lim_{x\to x_0} f(x) = L$ exists, then $\ds\lim_{x\to x_0} \sqrt{f(x)} = \sqrt{L}$.

\medskip

\textbf{Proof:} I will use sequential characterization of limits to prove the claim. For \textbf{all} sequences $a_n$ s.t. \[\ds\lim_{n\to\infty} a_n = x_0\]

then

\[\ds\lim_{n\to\infty}f(a_n) = L\]

Let $b_n = f(a_n)$ then

\[\ds\lim_{n\to\infty}b_n = \ds\lim_{n\to\infty}f(a_n) = L\]

Observe $\ds\lim_{n\to\infty}\sqrt{b_n}$. We know $\ds\lim_{n\to\infty}b_n$ exists, so by the square root property of sequences

\[\ds\lim_{n\to\infty}\sqrt{b_n} = \sqrt{\ds\lim_{n\to\infty}b_n} = \sqrt{L}\]

Therefore, if $f(x) \geq 0$ for all $x\in(a,b)$ and $\ds\lim_{x\to x_0} f(x) = L$ exists, then $\ds\lim_{x\to x_0} \sqrt{f(x)} = \sqrt{L}$.

\qedsym

\section*{5}

\textbf{Claim:} Let $f,g:(a, b)\rightarrow\R$ and let $x_0\in(a, b)$. If $\ds\lim_{x\to x_0} f(x) = L$ and $\ds\lim_{x\to x_0} g(x) = M$, then $\ds\lim_{x\to x_0} f(x)g(x) = LM$.

\medskip

\textbf{Proof:} I will use sequential characterization of limits to prove the claim. For \textbf{all} sequences $a_n$ s.t.

\[\ds\lim_{n\to\infty} a_n = x_0\]

then 

\[\ds\lim_{n\to\infty}f(a_n) = L\]

and

\[\ds\lim_{n\to\infty}g(a_n) = M\]

Let $b_n = f(a_b)$ and $c_n = g(a_n)$ then

\[\ds\lim_{n\to\infty}b_n = \ds\lim_{n\to\infty}f(a_n) = L\]

and

\[\ds\lim_{n\to\infty}b_n = \ds\lim_{n\to\infty}f(a_n) = L\]

Observe $\ds\lim_{n\to\infty}b_nc_n$. We know $\ds\lim_{n\to\infty}b_n$ exists and $\ds\lim_{n\to\infty}c_n$ exists, so by the product property of sequences

\[\ds\lim_{n\to\infty}b_nc_n = \ds\lim_{n\to\infty}b_n \cdot \ds\lim_{n\to\infty}c_n = LM\]

Therefore if $\ds\lim_{x\to x_0} f(x) = L$ and $\ds\lim_{x\to x_0} g(x) = M$, then $\ds\lim_{x\to x_0} f(x)g(x) = LM$.

\qedsym

\end{document}
