\documentclass[10pt,letterpaper]{article}

\usepackage{amsmath}
\usepackage{amsthm}
\usepackage{amssymb}
\usepackage{amsfonts}
\usepackage{array}

\newcommand\Q{\mathbf{Q}}
\newcommand\Z{\mathbf{Z}}
\newcommand\R{\mathbf{R}}
\newcommand\C{\mathbf{C}}
\newcommand\A{\mathbf{A}}
\newcommand\ds{\displaystyle}

\newcommand{\imp}{\Rightarrow}
\newcommand\qedsym{\hfill \rule{2mm}{2mm}}

\parindent=0pt
\begin{document}

\title{MATH5210 \textsc{Analysis}
  \\ Assignment 5
  \\ Limits of Functions
  \\ Philip Nelson
}

\date{}

\maketitle

\section*{1} 

\textbf{Lemma 1.1:} If $\alpha$ is a number s.t. $0\leq\alpha<\epsilon$, then $\alpha = 0$

\medskip

\textbf{Proof:} Assume that $\alpha > 0$, let $\epsilon = \frac{\alpha}{2}$.

Then \[\alpha < \frac{\alpha}{2}\] This conclusion is not possible, then the assumption must be incorrect. Therefore $\alpha = 0$.

\medskip

\textbf{Claim:} Let $f:(a,b)\to\R$ and let $x_0\in(a,b)$. If $\ds\lim_{x\to x_0} f(x)$ exists, then the limit is unique.

\medskip

\textbf{Proof:} Let $\epsilon > 0$ be given.

Then $\exists\ \delta_1$ s.t. \[|f(x) - L| < \frac{\epsilon}{2}\] for all $x$ s.t. \[0 < |x-x_0| < \delta_1\]

Likewise $\exists\ \delta_2$ s.t. \[|f(x) - M| < \frac{\epsilon}{2}\] for all $x$ s.t. \[0 < |x-x_0| < \delta_2\]

Then for $\delta = \min\{\delta_1, \delta_2\}$,

\[|L-M| = |L-f(x) + f(x) - B|\]

and then by the triangle inequality

\[|L-f(x) + f(x) - B| < |L - f(x)| + |f(x) - B|\]

which is equal to

\[=|\frac{\epsilon}{2}|+|\frac{\epsilon}{2}|\]

Since $\epsilon > 0$ was given

\[=\frac{\epsilon}{2}+\frac{\epsilon}{2} = \epsilon\]

So $0 \leq A-B < \epsilon$ and by lemma 1.1,

\[A-B = 0\]

\[\Rightarrow A=B\]

Therefore if $\ds\lim_{x\to x_0} f(x)$ exists, then the limit is unique. \qedsym

\section*{2}

\textbf{Claim:} Let $f:(a, b)\rightarrow\R$ and let $x_0\in(a, b)$. If $\ds\lim_{x\to x_0} f(x) = L$ and $L>0$, then $\exists\ \alpha, \delta>0$ s.t. $f(x)>\alpha\ \forall\ x\in(x_0-\delta, x_0+\delta)$.

\medskip

\textbf{Proof:} Since the limit exists, this is true for $\epsilon = \frac{L}{3},\ \exists\ \delta_1$ s.t. $0<|x-x_0| < \delta_1$.

Using the definition of convergence,
\[|f(x) - L| < \frac{L}{3}\]

\[\Rightarrow -\frac{L}{3} < f(x) - L < \frac{L}{3}\]

\[\Rightarrow L-\frac{L}{3} < f(x) < L+\frac{L}{3}\]

\[\Rightarrow \frac{2}{3}L < f(x) < \frac{4}{3}L\]

So let $\alpha = \frac{2}{3}L$, then for all $x$ s.t. $0 < |x-x_0| < \delta_1$

\[0 < \alpha = \frac{2}{3}L < f(x)\]

Therefore, if $\ds\lim_{x\to x_0} f(x) = L$ and $L>0$, then $\exists\ \alpha, \delta>0$ s.t. $f(x)>\alpha\ \forall\ x\in(x_0-\delta, x_0+\delta)$.

\qedsym

\section*{4}

\textbf{Claim:} Let $f:(a, b)\rightarrow\R$ and let $x_0\in(a, b)$. If $f(x) \geq 0$ for all $x\in(a,b)$ and $\ds\lim_{x\to x_0} f(x) = L$ exists, then $\ds\lim_{x\to x_0} \sqrt{f(x)} = \sqrt{L}$.

\medskip

\textbf{Proof:} I will use sequential characterization of limits to prove the claim. For \textbf{all} sequences $a_n$ s.t. \[\ds\lim_{n\to\infty} a_n = x_0\]

then

\[\ds\lim_{n\to\infty}f(a_n) = L\]

Let $b_n = f(a_n)$ then

\[\ds\lim_{n\to\infty}b_n = \ds\lim_{n\to\infty}f(a_n) = L\]

Observe $\ds\lim_{n\to\infty}\sqrt{b_n}$. We know $\ds\lim_{n\to\infty}b_n$ exists, so by the square root property of sequences

\[\ds\lim_{n\to\infty}\sqrt{b_n} = \sqrt{\ds\lim_{n\to\infty}b_n} = \sqrt{L}\]

Therefore, if $f(x) \geq 0$ for all $x\in(a,b)$ and $\ds\lim_{x\to x_0} f(x) = L$ exists, then $\ds\lim_{x\to x_0} \sqrt{f(x)} = \sqrt{L}$.

\qedsym

\section*{7}

\textbf{Preliminary Work:}

If $|x-2| < 1$

$\Rightarrow -1 < x-2 < 1$

$\Rightarrow 1 < x < 3$

$\Rightarrow 4 < x+3 < 6$

\medskip

\textbf{Claim:} $\ds\lim_{x\to 2} x^2 + x - 5 = 1$

\medskip

\textbf{Proof:} Let $\epsilon > 0$ be given. Choose $\delta = \min\{\frac{1}{6}\epsilon,1\}$ then if $|x-2| < \delta$

\[\Rightarrow |x-2| < 1\]

Then \[|f(x) - L| < \epsilon\]

\[\Rightarrow |x^2 + x - 5 - 1| < \epsilon\]

\[\Rightarrow |x-2| \cdot |x+3| < \epsilon\]

and by the preliminary work we know that for $|x-2|<1 \Rightarrow 4 < x+3 < 6$ so

\[\delta \cdot 6 < \epsilon\]

Therefore $\ds\lim_{x\to 2} x^2 + x - 5 = 1$

\qedsym

\section*{8}

\textbf{Preliminary Work:}

If $|x-1| < 1$

$\Rightarrow -1 < x-1 < 1$

$\Rightarrow 0 < x < 2$

$\Rightarrow 0 < x^2 < 4$

$\Rightarrow 4 < x^2+4 < 8$

$\Rightarrow 20 < 5(x^2+4) < 40$

and

If $|x-1| < 1$

$\Rightarrow -1 < x-1 < 1$

$\Rightarrow 0 < x < 2$

$\Rightarrow -4 < x-4 < -2$

$\Rightarrow |x-4| < 4$

\medskip

\textbf{Claim:} $\ds\lim_{x\to 1}\frac{x}{x^2+4} = \frac{1}{5}$

\medskip

\textbf{Proof:} Let $\epsilon > 0$ be given. Choose $\delta = \min\{5\epsilon, 1\}$ then if $|x-1| < \delta$

\[\Rightarrow |x-1| < 1\]

Then \[|f(x) - L| < \epsilon\]

\[\Rightarrow \left|\frac{x}{x^2+4} - \frac{1}{5}\right| < \epsilon\]

\[\Rightarrow \left|\frac{5x - x^2 - 4}{5x^2+20}\right| < \epsilon\]

\[\Rightarrow  \left|\frac{-(x-1)\cdot(x-4)}{5(x^2+4)}\right| < \epsilon\]

\[\Rightarrow  \frac{|x-1|\cdot|x-4|}{|5(x^2+4)|} < \epsilon\]

and by the preliminary work we know that for $|x-1| < 1\Rightarrow 20 < 5(x^2+4) < 40$ and $\Rightarrow |x-4| < 4$ so

\[\frac{|x-1|\cdot(4)}{20}<\epsilon\]

\[\Rightarrow \frac{\delta}{5}<\epsilon\]

Therefore $\ds\lim_{x\to 1}\frac{x}{x^2+4} = \frac{1}{5}$

\qedsym
 
\end{document}
